\documentclass[10pt, twocolumn, a4paper]{article}

\input{format.tex}  

\usepackage{amsmath}
\usepackage{amsfonts}
\usepackage{amssymb}
\usepackage{listings}

%----------------------------------------------------------------------------------------

\title{Report Example}

\author{
    \authorstyle{Alex Rogers, 29th October 2020}
}

%----------------------------------------------------------------------------------------

\date{}

\begin{document}

\maketitle  

\setcounter{page}{1}

\thispagestyle{firstpage}  

%----------------------------------------------------------------------------------------

\section{Overview}

This document provides a template for you to use to write your reports and also an example of the description of a model for the Millikan oil drop experiment. The modelling approaches are described further in the two course books \citep{bayesian_hackers, bayesian_analysis}.

\section{Millikan's Oil Drop Experiment}

The experimental set-up of Millikan's original oil drop experiment is shown in Figure \ref{experiment}.

\begin{figure}[htp]
\begin{center}
\includegraphics[width=0.475\textwidth]{./figures/experiment.jpg}
\caption{Millikan's oil drop experiment.}
\label{experiment}
\end{center}
\end{figure}

\subsection{Model Description}

We assume that we have $N$ measurements of the charge on $N$ individual oil drops. We denote this measurement as $c_i$ for $i \in \{0, \dots, N\}$. We assume that each measurement is affected by independent Gaussian noise such that:
\begin{equation}
c_i \sim \text{Normal}(n_i \times e, \sigma^2)
\end{equation}
where $n_i$ is the number of additional electrons on each oil drop and $e$ is the charge on an electron. We do not directly observe $n_i$ and thus we assign it a Poisson prior given by:
\begin{equation}
n_i \sim 1 + \text{Poisson}(1)
\end{equation}
Similarly, we model $\sigma$ and $e$ with appropriate priors such that:
\begin{align}
\sigma & \sim \text{Exponential}(0.1) \\
e & \sim \text{Exponential}(1) 
\end{align}

\subsection{Solving the Model}

We solve the model using PyMC.\footnote{\url{https://docs.pymc.io}} Figure \ref{results} shows the posterior probability distribution for the charge on the electron and the accuracy of the experimental measurements.

\begin{figure}[tp]
\begin{center}
\includegraphics[width=0.475\textwidth]{./figures/results.pdf}
\caption{Posterior probability density functions for the charge on the electron ($e$) and the measurement error ($\sigma$).}
\label{results}
\end{center}
\end{figure}

\printbibliography[title={References}] 

\end{document}





